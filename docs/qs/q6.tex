\question Vinte e seis trabalhadores de plataformas de petr´oleo offshore participaram de um exercício de fuga simulado, resultando nos dados a seguir (em segundos) para concluir a fuga:
\begin{parts}
    \part  Construa um diagrama de ramos-e-folhas dos dados. Como ele sugere que a média e mediana serão comparadas?
    \begin{solution}
        \begin{table}[H]
            \centering
            \begin{tabular}{r|l@{\hspace{4 pt}}l@{\hspace{4 pt}}l@{\hspace{4 pt}}l@{\hspace{4 pt}}l@{\hspace{4 pt}}}
                32 & 5 & 5             \\
                33 & 4 & 9             \\
                34                     \\
                35 & 6 & 6 & 9 & 9     \\
                36 & 3 & 4 & 4 & 6 & 9 \\
                37 & 0 & 3 & 3 & 4 & 5 \\
                38 & 9                 \\
                39 & 2 & 3 & 4 & 7     \\
                40 & 2 & 3             \\
                41                     \\
                42 & 4                 \\
            \end{tabular}
        \end{table}

        Assim teremos:

        \begin{equation*}
            \begin{split}
                \overline{X} & = 370.6923 \\
                M_d          & = 369.5
            \end{split}
        \end{equation*}
    \end{solution}

    \part   Calcule os valores da média ($\overline{x}$) e da mediana ($M_d$) amostrais. Dica: $\sum{xi} = 9638$.
    \begin{solution}
        Primeiro é necessário fazer a ordenação desses valores, assim teremos:
        \begin{equation*}
            \begin{split}
                \overline{x} & = \frac{\sum{x_i}}{n} = \frac{9638}{26} = 370.6923 \\
            \end{split}
        \end{equation*}
        Dado a presença de 26 observações, um número par, temos que a mediana será a média dos dois valores centrais:
        \begin{equation*}
            \begin{split}
                M_d & = \frac{x_{13} + x_{14}}{2} = \frac{369 + 370}{2} = 369.5
            \end{split}
        \end{equation*}
    \end{solution}

    \part Em quantos segundos o maior tempo, atualmente 424, pode ser aumentado sem afetar o valor da mediana amostral \footnote{Valor que ocupa posição central na amostra ordenada}?
    \begin{solution}
        \begin{equation*}
            \begin{split}
                M_d & = \begin{cases}
                            x_{(\frac{n+1}{2})}                                 & \text{se n ímpar} \\
                            \frac{x_{(\frac{n}{2})} + x_{(\frac{n}{2} + 1)}}{2} & \text{se n par}
                        \end{cases}
            \end{split}
        \end{equation*}
        Logo, não afetará a mediana se um valor muito grande for apresentado nos extremos da amostra.
    \end{solution}

    \part Quais são os valores de $\overline{x}$ e $M_d$ quando as observações são expressas em minutos?
    \begin{solution}
        Ao transformar todos os valores em minutos, teremos:
        \begin{equation*}
            \begin{split}
                \overline{X} & = 6.178077 \\
                M_d          & = 6.155
            \end{split}
        \end{equation*}
    \end{solution}
\end{parts}
