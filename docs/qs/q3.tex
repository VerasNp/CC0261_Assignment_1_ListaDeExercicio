\question Em uma padaria foi feita uma pesquisa para verificar o consumo de leite e de pão nos primeiros dez dias do mês de janeiro. Foram levantados os seguintes valores diários:
\begin{table}[H]
    \centering
    \begin{tabular}{|c|c|c|c|c|c|c|c|c|c|c|}
        \hline
        Consumo de leite (litros) & 25 & 26 & 30 & 30 & 28 & 23 & 25 & 29 & 34 & 30 \\
        \hline
        Consumo de pão (Kg)       & 31 & 40 & 36 & 39 & 39 & 40 & 42 & 38 & 39 & 41 \\
        \hline
    \end{tabular}
\end{table}
\begin{parts}
    \part Em média, qual foi o produto mais consumido nesses dias?
    \begin{solution}
        \begin{equation*}
            \begin{aligned}
                \overline{X}_{leite} & = \frac{\sum_{n}^{i = i}}{n}                                      \\
                                     & = \frac{25 + 26 + 30 + 30 + 28 + 23 + 25 + 29 + 34 + 30}{10} = 28 \\
            \end{aligned}
        \end{equation*}
        \begin{equation*}
            \begin{aligned}
                \overline{X}_{p\tilde{a}o} & = \frac{\sum_{n}^{i = i}}{n}                                        \\
                                           & = \frac{31 + 40 + 36 + 39 + 39 + 40 + 42 + 38 + 39 + 41}{10} = 38.5 \\
            \end{aligned}
        \end{equation*}

        Assim podemos ver que, em média o consumo de pão foi maior.
    \end{solution}
    \part Qual dos produtos teve maior variação no consumo\footnote{Dado a diferença de unidades de medida usadas, a comparação direta não seria adequada, assim utilizamos o \textbf{Coeficiente de Variação (CV)}} , justifique sua resposta.
    \ifprintanswers
    \savenotes 
    \fi
    \begin{solution}
        A variação será dada através do \textbf{coeficiente de variação (CV)} \footnote{Dado pela razão entre o desvio padrão $s$ e a média amostral (normalmente expresso em porcentagens)} tendo a fórmula:
        \begin{equation*}
            \begin{aligned}
                CV           & = \frac{s}{\overline{X}} \times 100;                          \\
                s            & = \sqrt{\frac{\sum_{n}^{i = 1}(X_{i} - \overline{X})^{2}}{n}} \\
                \overline{X} & = \frac{\sum_{n}^{i = i}}{n}
            \end{aligned}
        \end{equation*}

        Assim temos:

        \begin{equation*}
            \begin{aligned}
                s_{leite} = 3.265986 \ \overline{X} = 28 \therefore    \\ CV_{leite} = \frac{3.265986}{28} \times 100 = 11.66424\% \approx 11.66\%  \\
                s_{p\tilde{a}o} = 3.100179 \ \overline{X} = 38.5 \therefore \\ CV_{p\tilde{a}o} = \frac{3.100179}{38.5} \times 100 = 8.051948\% \approx 8.05\%
            \end{aligned}
        \end{equation*}

        Dessa forma temos um CV maior para o leite, logo o \textbf{leite teve maior variação no consumo}.
    \end{solution}
    \ifprintanswers
    \spewnotes
    \fi
\end{parts}