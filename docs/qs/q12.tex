\question O número de divórcios em uma determinada cidade, de acordo com a duração do casamento, está representado na tabela abaixo.
\begin{parts}
    \part Qual a duração média dos casamentos? E a duração mediana?
    \begin{solution}
        Para o cálculo da média ($\overline{x}$) façamos:
        \begin{equation*}
            \begin{split}
                \overline{x} & = \frac{\sum^{k}_{i = 1}{n_ix_i}}{n}, \text{onde } x_i \text{ é a média da classe $i$}            \\
                \overline{x} & = \frac{\sum^{5}_{i = 1}{n_ix_i}}{5000} = \frac{(2800 \times 3) + \dots + (50 \times 27)}{5000} = \\
                \overline{x} & = 6.9
            \end{split}
        \end{equation*}
        Para o cálculo da mediana ($M_d$), será necessário informações como frequência relativa e relativa acumulada, assim:

        % latex table generated in R 4.3.3 by xtable 1.8-4 package
        % Tue Apr  9 01:35:51 2024
        \begin{table}[H]
            \centering
            \begin{tabular}{rrrr}
                \hline
                Classe & freqAcum ($N$) & freqReq ($f$) & freqRelAcum ($F$) \\
                \hline
                1      & 2800           & 0.56          & 0.56              \\
                2      & 4200           & 0.28          & 0.84              \\
                3      & 4800           & 0.12          & 0.96              \\
                4      & 4950           & 0.03          & 0.99              \\
                5      & 5000           & 0.01          & 1.00              \\
                \hline
            \end{tabular}
        \end{table}
        Ao analisar a tabela das frequências é possível identificar como a classe da mediana sendo $0 \vdash 6 (n_i = 2800)$, então:
        \begin{equation*}
            \begin{split}
                M_d & = l_i + \frac{h(0.5 - F_{i-1})}{f_i} = \\
                    & = 0 + \frac{6(0.5 - 0)}{0.56} =        \\
                M_d & \approx 5.36
            \end{split}
        \end{equation*}
    \end{solution}

    \part Encontre a variância e o desvio-padrão da duração dos casamento.
    \begin{solution}
        Para o cálculo da variância ($s^2$) teremos:
        \begin{equation*}
            \begin{split}
                s^2 & = \frac{\sum^{k}_{i = 1}{n_i(x - \overline{x})^2}}{n-1} = \frac{1}{n-1} \times \Bigg[ \sum^{k}_{i = 1}{n_ix_i^2} - n\overline{x}^2 \Bigg] \\
                    & = \frac{1}{4999} \times \Bigg[ (2800 \times 3^2) + \dots + (50 \times 27^2) - 5000 \times 6.9^2 \Bigg]                                    \\
                s^2 & = 27.64
            \end{split}
        \end{equation*}
        O desvio-padrão será dado por $s = \sqrt{s^2} = \sqrt{27.64} \approx 5.26$
    \end{solution}

    \part Encontre o 1º e 9º decil. Interprete.
    \begin{solution}
        Nesse caso é pedido o 1º ($Q_{0.10}$) e 9º ($Q_{0.90}$) decis, para isso utilizaremos a seguinte fórmula (Válida para dados agrupados):
        \begin{equation*}
            \begin{split}
                Q_p & = l_i + \frac{h(p - F_{i-1})}{f_i}
            \end{split}
        \end{equation*}
        Será útil em nossos cálculos informações como frequência relativa e relativa acumulada, assim:
        % latex table generated in R 4.3.3 by xtable 1.8-4 package
        % Tue Apr  9 01:35:51 2024
        \begin{table}[H]
            \centering
            \begin{tabular}{rrrr}
                \hline
                Classe & freqAcum ($N$) & freqReq ($f$) & freqRelAcum ($F$) \\
                \hline
                1      & 2800           & 0.56          & 0.56              \\
                2      & 4200           & 0.28          & 0.84              \\
                3      & 4800           & 0.12          & 0.96              \\
                4      & 4950           & 0.03          & 0.99              \\
                5      & 5000           & 0.01          & 1.00              \\
                \hline
            \end{tabular}
        \end{table}
        Então:
        \begin{equation*}
            \begin{split}
                Q_{0.10} & = 0 + \frac{6(0.10 - 0)}{0.56} = 1.07 \\
                Q_{0.90} & = 0 + \frac{6(0.90 - 0.56)}{0.28} = 5.57
            \end{split}
        \end{equation*}
    \end{solution}

    \part Qual o desvio interquartílico?
    \begin{solution}
        O cálculo do desvio interquartílico ($dq$) é dado por $dq = Q_3 - Q_1$, dessa forma temos:
        \begin{equation*}
            \begin{split}
                Q_1 = Q_{0.25} & = 0 + \frac{6(0.25 - 0)}{0.56} = 2.68 \\
                Q_3 = Q_{0.75} & = 0 + \frac{6(0.75 - 0)}{0.56} = 8.04  \\
                dq & = Q_3 - Q_1 = 8.04 - 2.68 = 5.36
            \end{split}
        \end{equation*}
    \end{solution}
\end{parts}