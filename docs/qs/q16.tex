\question Em uma granja foi observada a distribuição dos frangos em relação ao peso, que era a
seguinte

\begin{parts}
    \part Qual a média da distribuição?
    \begin{solution}
        Por se tratar de uma \textbf{distribuição por classe}, temos que a média é dada por:
        \begin{equation*}
            \begin{split}
                \overline{X} & = \frac{\sum\limits^{k}_{i=1}{n_i \times x_i}}{n}, \text{onde } x_i \text{ é} \textbf{ média da classe modal.}                     \\
                             & = \frac{(60 \times 970) + (160 \times 990) + (280 \times 1010) + (260 \times 1030) + (160 \times 1050) + (80 \times 1070)}{1000} = \\
                \overline{X} & = \frac{1020800}{1000} = 1020.8
            \end{split}
        \end{equation*}
    \end{solution}

    \part Qual a variância\footnote{\href{https://pt.khanacademy.org/math/ap-statistics/summarizing-quantitative-data-ap/more-standard-deviation/v/another-simulation-giving-evidence-that-n-1-gives-us-an-unbiased-estimate-of-variance}{Motivo pelo qual dividimos por $n - 1$.}} da distribuição?
    \begin{solution}
        \begin{equation*}
            \begin{split}
                S^2 & = \frac{1}{n - 1} \Bigg[\sum\limits^{k}_{i = 1}{n_i(x_i - \overline{x})^2}\Bigg], \text{onde } x_i \text{ é} \textbf{ média da classe modal.} \\
            \end{split}
        \end{equation*}
    \end{solution}

    \part Queremos dividir os frangos em quatro categorias, em relação ao peso, de modo que:
    \begin{itemize}
        \item os $20\%$ mais leves sejam da categoria D;
        \item os $30\%$ seguintes sejam da categoria C;
        \item os $30\%$ seguintes sejam da categoria B;
        \item os $20\%$ seguintes (ou seja, os $20\%$ mais pesados) sejam da categoria A.
    \end{itemize}
    Quais  os limites de peso entre as categorias A, B, C e D?
    \begin{solution}
        Essa pergunta está relacionada a parte de \textbf{quantis}, para facilitar teremos de calcular a frequência relativa e acumulada, o que teremos:
        % latex table generated in R 4.3.3 by xtable 1.8-4 package
        % Sun Apr  7 21:18:50 2024
        \begin{table}[H]
            \centering
            \begin{tabular}{rrr}
                \hline
                  & FreqRel & FreqAcum \\
                \hline
                1 & 0.06    & 0.06     \\
                2 & 0.16    & 0.22     \\
                3 & 0.28    & 0.50     \\
                4 & 0.26    & 0.76     \\
                5 & 0.16    & 0.92     \\
                6 & 0.08    & 1.00     \\
                \hline
            \end{tabular}
        \end{table}
        Para encontrar os limites de peso vamos utilizar a fórmula de quantis em distribuições com classe, isto é:
    \end{solution}

    \part O granjeiro decide separar desse lote os animais com peso inferior a dois desvios padrões abaixo da média para receberem ração reforçada, e também separar os ani- mais com peso superior a um e meio desvio padrão acima da média para usá-los como reprodutores. Qual a porcentagem de animais que serão separados em cada caso?
    \begin{solution}
        ?
    \end{solution}
\end{parts}