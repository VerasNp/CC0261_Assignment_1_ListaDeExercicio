\question Identifique cada uma das variáveis seguintes como qualitativa, quantitativa e como ordinal, nominal ou discreta, contínua.
\begin{parts}
    \part A concentração de impurezas em uma amostra de leite, em mg/l.
    \begin{solution}
        É uma variável quantitativa, como não é possível colocar os valores em um conjunto enumerável (dado que uma medida poderia apresentar valor 1 ou então 1,0000001) se trata de uma variável quantitativa contínua.
    \end{solution}
    \part Tipo de escola de cada candidato ao vestibular da UFC em um determinado ano.
    \begin{solution}
        Esta pode ser classificada como uma variável qualitativa. Como temos classificações de escolas como sendo pública ou privadas (sem uma ordem estabelecida entre esses valores), esta é uma variável qualitativa nominal.
    \end{solution}
    \part O número de moradores em cada residência de uma cidade.
    \begin{solution}
        Se trata se uma variável quantitativa, e como podemos colocar os valores encontrados em um conjunto enumerável se trata de uma variável quantitativa discreta.
    \end{solution}
    \part A temperatura de certa região, em determinada época do ano.
    \begin{solution}
        A temperatura pode tomar qualquer valor em uma dada escala, desde valores inteiros como 30 graus até valores fracionários 30,6 graus, por exemplo. Portanto, é uma variável quantitativa contínua.
    \end{solution}
    \part A produção por hectare de determinado tipo de grão.
    \begin{solution}
        A produção por hectare pode tomar qualquer valor em uma dada escala, desde valores inteiros como 30 kg/ha até valores fracionários 30,6 kg/ha, por exemplo. Portanto, é uma variável quantitativa contínua.
    \end{solution}
\end{parts}